\let\negmedspace\undefined
\let\negthickspace\undefined
\documentclass[journal]{IEEEtran}
\usepackage[a5paper, margin=10mm, onecolumn]{geometry}
%\usepackage{lmodern} % Ensure lmodern is loaded for pdflatex
\usepackage{tfrupee} % Include tfrupee package

\setlength{\headheight}{1cm} % Set the height of the header box
\setlength{\headsep}{0mm}     % Set the distance between the header box and the top of the text
\usepackage{multicol}
\usepackage{gvv-book}
\usepackage{gvv}
\usepackage{cite}
\usepackage{amsmath,amssymb,amsfonts,amsthm}
\usepackage{algorithmic}
\usepackage{graphicx}
\usepackage{textcomp}
\usepackage{xcolor}
\usepackage{txfonts}
\usepackage{listings}
\usepackage{enumitem}
\usepackage{mathtools}
\usepackage{gensymb}
\usepackage{comment}
\usepackage[breaklinks=true]{hyperref}
\usepackage{tkz-euclide} 
\usepackage{listings}
% \usepackage{gvv}                                        
\def\inputGnumericTable{}                                 
\usepackage[latin1]{inputenc}                                
\usepackage{color}                                            
\usepackage{array}                                            
\usepackage{longtable}                                       
\usepackage{calc}                                             
\usepackage{multirow}                                         
\usepackage{hhline}                                           
\usepackage{ifthen}                                           
\usepackage{lscape}
\usepackage{tikz}
\begin{document}

\bibliographystyle{IEEEtran}
\vspace{3cm}

\title{2007-PH}
\author{EE24BTECH11027 - satwikagv}
% \maketitle
% \newpage
% \bigskip
{\let\newpage\relax\maketitle}

\renewcommand{\thefigure}{\theenumi}
\renewcommand{\thetable}{\theenumi}
\setlength{\intextsep}{10pt} % Space between text and floats


\numberwithin{equation}{enumi}
\numberwithin{figure}{enumi}
\renewcommand{\thetable}{\theenumi}
\begin{enumerate}
	\item An electromagnetic wave with $\vec{E}\brak{z,t}=E_0\cos\brak{\omega t-kz}\hat{i}$ is traveling in free space and crosses a disc of radius 2 m placed perpendicular to the z-axis. If $E_0=60 V m^{-1}$, the average power, in Watt, crossing the disc along the $z$-direction is 
\begin{multicols}{4}
\begin{enumerate}
    \item 30
    \item 60
    \item 120
    \item 270
\end{enumerate}
\end{multicols}
\item Can the following scalar and vector potentials describe an electromagnetic field? 
$\phi\brak{\bar{x},t}=3xyz-4t$ \\
$\vec{A}\brak{\bar{x},t}=\brak{2x-\omega t}\hat{i}+\brak{y-2z}\hat{j}+\brak{z-2xe^{i\omega t}}\hat{k}$
\\ where $\omega$ is a constant.
\begin{multicols}{2}
    \begin{enumerate}
        \item Yes, in the Coulomb gauge.
        \item Yes, in the Lorentz gauge.
        \item Yes, provided $\omega = 0$.
        \item No.
    \end{enumerate}
\end{multicols}
\item For a particle of mass $m$ in a one-dimensional harmonic oscillator potential of the form $V\brak{x}=\frac{1}{2}m{\omega}^2x^2$, the first excited energy eigenstate is $\psi \brak{x}=xe^{-ax^2}$. The value of $a$ is
\begin{multicols}{4}
    \begin{enumerate}
	    \item $\frac{m\omega}{4\hbar}$
	    \item $\frac{m\omega}{3\hbar}$
	    \item $\frac{m\omega}{2\hbar}$
	    \item $\frac{2m\omega}{3\hbar}$
    \end{enumerate}
\end{multicols}
\item If $\sbrak{x,p}=i\hbar$, the value of \sbrak{x^3,p} is
\begin{multicols}{4}
    \begin{enumerate}
        \item $2i\hbar x^2$
        \item $-2i\hbar x^2$
        \item $3i\hbar x^2$
        \item $-3i\hbar x^2$
    \end{enumerate}
\end{multicols}
\item There are only three bound states for a particle mass $m$ in a one-dimensional potential well of the form shown in the figure. The depth $V_0$ of the potential satisfies
\begin{minipage}[t]{0.5\textwidth}
\raggedleft
\resizebox{0.6\textwidth}{!}{
\begin{tikzpicture}
\draw[->] (-5,0) -- (5,0) node[right] {\Large $\textbf{x}$};
\draw[->] (0,-3) -- (0,2) node[above] {\huge $\textbf{V}$};
\node at (-2.5, -0.2) [above] {\Huge \textbf{$-\frac{a}{2}$}};  
\node at (2.5, -0.2) [above] {\Huge \textbf{$\frac{a}{2}$}};   
\node at (0.2, -2) [below] {\Huge \textbf{$-V_0$}};  
\draw[thick] (-5,0) -- (-2.5,0);
\draw[thick] (-2.5,0) -- (-2.5,-2); 
\draw[thick] (-2.5,-2) -- (2.5,-2);
\draw[thick] (2.5,-2) -- (2.5,0);
\draw[thick] (2.5,0) -- (5,0); 
\end{tikzpicture}
}
\end{minipage}
\begin{multicols}{2}
    \begin{enumerate}
        \item $\frac{2\pi^2 h^2}{ma^2}<V_0<\frac{9\pi^2 h^2}{ma^2}$
        \item $\frac{\pi^2 h^2}{ma^2}<V_0<\frac{2\pi^2 h^2}{ma^2}$
        \item $\frac{2\pi^2 h^2}{ma^2}<V_0<\frac{8\pi^2 h^2}{ma^2}$
        \item $\frac{2\pi^2 h^2}{ma^2}<V_0<\frac{50\pi^2 h^2}{ma^2}$
    \end{enumerate}
\end{multicols}
\item An atomic state of hydrogen is represented by the following wavefunction:\\
	$\psi\brak{r,\theta,\phi}=\frac{1}{\sqrt{2}}\brak{\frac{1}{a_0}}^{\frac{3}{2}}\brak{1-\frac{r}{2a_0}}e^{\frac{-r}{2a_0}}\cos\theta$.\\ where $a_0$ is a constant. The quantum numbers of the state are
\begin{multicols}{2}
    \begin{enumerate}
        \item $l=0,m=0,n=1$
        \item $l=1,m=1,n=2$
        \item $l=1,m=0,n=2$
        \item $l=2,m=0,n=3$
    \end{enumerate}
\end{multicols}
\item  Three operators $X,Y$ and $Z$ satisfy the commutation relations\\ $\sbrak{X,Y}=i\hbar Z,\sbrak{Y, Z}=i\hbar X,\sbrak{Z, X}=i\hbar Y$ .\\The set of all possible eigenvalues of the operator $Z$, in units of $\hbar$, is
\begin{multicols}{2}
    \begin{enumerate}
        \item \cbrak{0,\pm 1,\pm 2,\pm 3,\dots}
        \item \cbrak{\frac{1}{2},1,\frac{3}{2},2,\frac{5}{2},\dots}
        \item \cbrak{0,\pm\frac{1}{2},\pm 1,\pm\frac{3}{2},\pm 2,\pm\frac{5}{2},\dots}
        \item \cbrak{-\frac{1}{2},\frac{1}{2}}
    \end{enumerate}
\end{multicols}
\item A heat pump working on the Carnot cycle maintains the inside temperature of a house at 22\degree C by supplying $450 kJ s^{-1}$. If the outside temperature is 0\degree C, the heat taken, in $kJ s^{-1}$, from the outside air is approximately
\begin{multicols}{4}
    \begin{enumerate}
        \item 487
        \item 470
        \item 467
        \item 417
    \end{enumerate}
\end{multicols}
\item The vapour pressure $p$ \brak{\text{in mm of Hg}} of a solid, at temperature $T$, is expressed by $\ln p = 23-\frac{3863}{T}$ and that of its liquid phase by $\ln p = 19-\frac{3063}{T}$. The triple point\brak{\text{in Kelvin}} of the material is
\begin{multicols}{4}
    \begin{enumerate}
        \item 185
        \item 190
        \item 195
        \item 200
    \end{enumerate}
\end{multicols}
\item The free energy of a photon gas is given by $F=-\brak{\frac{a}{3}}VT^4$, where $a$ is a constant. The entropy $S$ and the pressure $P$ of the photon gas are
\begin{multicols}{2}
    \begin{enumerate}
        \item $S=\frac{4}{3}aVT^4, P=\frac{a}{3}T^4$
        \item $S=\frac{1}{3}aVT^4, P=\frac{4a}{3}T^3$
        \item $S=\frac{4}{3}aVT^4, P=\frac{a}{3}T^3$
        \item $S=\frac{1}{3}aVT^4, P=\frac{4a}{3}T^4$
    \end{enumerate}
\end{multicols}
\item A system has energy levels $E_0,2E_0,3E_0,\dots$, where the excited states are triply degenerate. Four non interacting bosons is $5E_0$, the number of microstates is
\begin{multicols}{4}
    \begin{enumerate}
        \item 2
        \item 3
        \item 4
        \item 5
    \end{enumerate}
\end{multicols}
\item In accordance with the selection rules for electric dipole transitions, the $4\prescript{3}{}{P}_1$ state of helium can decay by photo emission to the states
\begin{multicols}{2}
    \begin{enumerate}
        \item $2\prescript{1}{}{S}_0,2\prescript{1}{}{P}_1 \text{ and } 3\prescript{1}{}{D}_2$
        \item $3\prescript{1}{}{S}_0,3\prescript{1}{}{P}_1 \text{ and } 3\prescript{1}{}{D}_2$
        \item $3\prescript{3}{}{P}_2,3\prescript{3}{}{P}_0 \text{ and } 3\prescript{3}{}{D}_3$
        \item $2\prescript{3}{}{S}_1,3\prescript{3}{}{D}_2 \text{ and } 3\prescript{3}{}{D}_1$
    \end{enumerate}
\end{multicols}
\item If an atom is in the $\prescript{3}{}{D}_3$ state, the angle between the its orbital and spin angular momentum vectors \brak{\vec{L}\text{ and }\vec{S}} is 
\begin{multicols}{4}
    \begin{enumerate}
        \item $\cos^{-1}\frac{1}{\sqrt{3}}$
        \item $\cos^{-1}\frac{2}{\sqrt{3}}$
        \item $\cos^{-1}\frac{1}{2}$
        \item $\cos^{-1}\frac{\sqrt{3}}{2}$
    \end{enumerate}
\end{multicols}
\item  The hyperfine structure of $Na\brak{3\prescript{2}{}{P}_\frac{3}{2}}$ with nuclear spin $I=\frac{3}{2}$ has 
\begin{multicols}{4}
   \begin{enumerate}
       \item 1 state 
       \item 2 states
       \item 3 states
       \item 4 states
   \end{enumerate}
\end{multicols}
\item The allowed rational energy levels of a rigid hetero-nuclear diatomic molecule are expressed as $\epsilon_j=BJ\brak{J+1}$, where $B$ is the rotational constant and $J$ is a rotational quantum number.\\ In a system of such diatomic molecules of reduced  mass $\mu$, some of the atoms of one element are replaced by a heavier isotope, such that the reduced mass is changed to $1.05\mu$. In the rotational spectrum of the system, the shift in the spectral line, corresponding to a transition $J=4 \rightarrow J=5$, is 
\begin{multicols}{4}
    \begin{enumerate}
        \item 0.475 $B$
        \item 0.50 $B$
        \item 0.95 $B$
        \item 1.0 $B$
    \end{enumerate}
\end{multicols}
\item The number of fundamental vibrational modes of $CO_2$ molecule is 
\begin{enumerate}
    \item four: 2 are Raman active and 2 are infrared active.
    \item four: 1 are Raman active and 3 are infrared active.
    \item three: 1 are Raman active and 2 are infrared active.
    \item three: 2 are Raman active and 1 are infrared active.
\end{enumerate}
\item A piece of paraffin is placed in a uniform magnetic field $H_0$. The sample contains hydrogen nuclei of mass $m_p$, which interact only with external magnetic field. An additional oscillating magnetic field is applied to observe resonance absorption. If $g_l$ is the $g$-factor of the hydrogen nucleus, the frequency, at which resonance absorption takes place, is given by
\begin{multicols}{4}
    \begin{enumerate}
        \item $\frac{3g_leH_0}{2\pi m_p}$
        \item $\frac{3g_leH_0}{4\pi m_p}$
        \item $\frac{g_leH_0}{2\pi m_p}$
        \item $\frac{g_leH_0}{4\pi m_p}$
    \end{enumerate}
\end{multicols}
\end{enumerate}
\end{document}
