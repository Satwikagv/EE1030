\let\negmedspace\undefined
\let\negthickspace\undefined
\documentclass[journal]{IEEEtran}
\usepackage[a5paper, margin=10mm, onecolumn]{geometry}
%\usepackage{lmodern} % Ensure lmodern is loaded for pdflatex
\usepackage{tfrupee} % Include tfrupee package

\setlength{\headheight}{1cm} % Set the height of the header box
\setlength{\headsep}{0mm}     % Set the distance between the header box and the top of the text
\usepackage{multicol}
\usepackage{gvv-book}
\usepackage{gvv}
\usepackage{cite}
\usepackage{amsmath,amssymb,amsfonts,amsthm}
\usepackage{algorithmic}
\usepackage{graphicx}
\usepackage{textcomp}
\usepackage{xcolor}
\usepackage{txfonts}
\usepackage{listings}
\usepackage{enumitem}
\usepackage{mathtools}
\usepackage{gensymb}
\usepackage{comment}
\usepackage[breaklinks=true]{hyperref}
\usepackage{tkz-euclide} 
\usepackage{listings}
% \usepackage{gvv}                                        
\def\inputGnumericTable{}                                 
\usepackage[latin1]{inputenc}                                
\usepackage{color}                                            
\usepackage{array}                                            
\usepackage{longtable}                                       
\usepackage{calc}                                             
\usepackage{multirow}                                         
\usepackage{hhline}                                           
\usepackage{ifthen}                                           
\usepackage{lscape}
\begin{document}

\bibliographystyle{IEEEtran}
\vspace{3cm}

\title{01-02-2023 s1}
\author{EE24BTECH11027-satwikagv}
% \maketitle
% \newpage
% \bigskip
{\let\newpage\relax\maketitle}

\renewcommand{\thefigure}{\theenumi}
\renewcommand{\thetable}{\theenumi}
\setlength{\intextsep}{10pt} % Space between text and floats


\numberwithin{equation}{enumi}
\numberwithin{figure}{enumi}
\renewcommand{\thetable}{\theenumi}

\begin{enumerate}
	\item \begin{align*}\lim_{n \to \infty} \brak{ \frac{1}{1+n} + \frac{1}{2+n} + \frac{1}{3+n} + \dots +\frac{1}{2n} }\end{align*} is equal to :-
\begin{multicols}{2}
\begin{enumerate}
    \item 0
    \item $\log_e(2)$
    \item $\log_e(\frac{3}{2})$
    \item $\log_e(\frac{2}{3})$
\end{enumerate}
\end{multicols}
\item The negation of the expression $q \vee \brak{\brak{\neg q}\wedge p}$ is equivalent to 
\begin{multicols}{2}
\begin{enumerate}
\item $\brak{\neg p} \wedge \brak{\neg q}$
\item $p \wedge \brak{\neg q}$
\item $\brak{\neg p} \vee \brak{\neg q}$
\item $\brak{\neg p} \vee q$
\end{enumerate}
\end{multicols}
\item In a binomial distribution $\Vec{B}\brak{n,p}$, the sum and product of the mean and variance are 5 and 6 respectively,then find 6\brak{n+p-q} is equal to :-
\begin{multicols}{2}
\begin{enumerate}
    \item 51
    \item 52
    \item 53
    \item 50
\end{enumerate}
\end{multicols}
\item The sum to 10 terms of the series $\frac{1}{1+1^2+1^4}+\frac{2}{1+2^2+2^4}+\frac{3}{1+3^2+3^4}+ \dots$ is :-
\begin{multicols}{2}
\begin{enumerate}
    \item $\frac{59}{111}$
    \item $\frac{55}{111}$
    \item $\frac{56}{111}$
    \item $\frac{58}{111}$
\end{enumerate}
\end{multicols}
\item The value is $\frac{1}{1!50!}+\frac{1}{3!48!}+\frac{1}{5!46!}+\dots+\frac{1}{49!2!}+\frac{1}{51!1!}$ is 
\begin{multicols}{2}
\begin{enumerate}
\item $\frac{2^{50}}{50!}$
\item $\frac{2^{50}}{51!}$
\item $\frac{2^{51}}{51!}$
\item $\frac{2^{51}}{50!}$
\end{enumerate}
\end{multicols}
\item If the orthocentre of the triangle, whose vertices are \brak{1,2},\brak{2,3} and \brak{3,1} is \brak{\alpha,\beta}, then the quadratic equation whose roots are $\alpha +4\beta$ and $4\alpha +\beta$,is  
\begin{enumerate}
\item $x^2-19x+90=0$
\item $x^2-18x+80=0$
\item $x^2-22x+120=0$
\item $x^2-20x+99=0$
\end{enumerate}
\item For a triangle $ABC$, the value of $\cos{2A}+\cos{2B}+\cos{2C}$ is least.If its inradius is 3 and incentre is $M$, then which of the following is NOT correct? 
\begin{enumerate}
\item Perimeter of $\Delta ABC$ is $18\sqrt{3}$
\item $\sin{2A}+\sin{2B}+\sin{2C}=\sin{A}+\sin{B}+\sin{C}$
\item $\overrightarrow{MA}.\overrightarrow{MB}=-18$
\item area of $\Delta ABC$ is $\frac{27\sqrt{3}}{2}$
\end{enumerate}
\item The combined equation of the two lines $ax+by+c=0$ and $a^\prime x+b^\prime y+c^\prime=0$can be written as $\brak{ax+by+c}\brak{a^\prime x+b^\prime y+c^\prime}=0$.The equation of the angle bisectors of the lines represented by the equation $2x^2+xy-3y^2=0$ is 
\begin{enumerate}
\item $3x^2+5xy+2y^2=0$
\item $x^2-y^2+10xy=0$
\item $3x^2+xy-2y^2=0$
\item $x^2-y^2-10xy=0$    
\end{enumerate}
\item The shortest distance between the lines $\frac{x-5}{1}=\frac{y-2}{2}=\frac{z-4}{-3}$ and $\frac{x+3}{1}=\frac{y+5}{4}=\frac{z-1}{-5}$
\begin{multicols}{2}
\begin{enumerate}
\item $7\sqrt{3}$
\item $5\sqrt{3}$
\item $6\sqrt{3}$
\item $4\sqrt{3}$
\end{enumerate}
\end{multicols}
\item Let $S$ denote the set of all real values of $\lambda$ such that the system of equations
\begin{align*}
    \lambda x+y+z&=1\\
    x+\lambda y+z&=1\\
    x+y+\lambda z&=1
\end{align*}
is inconsistent, then $\displaystyle \sum_{\lambda \in S}\brak{\abs{\lambda^2} + \abs{\lambda}}$ is equal to 
\begin{enumerate}
\item 2
\item 12
\item 4
\item 6
\end{enumerate}
\item Let $S=\cbrak{ x:x \in \mathbb{R}  and  \brak{\sqrt{3}+\sqrt{2}}^{x^2-4}+\brak{ \sqrt{3}-\sqrt{2}}^{x^2-4}=10}.$ Then $n\brak{S}$ is equal to 
\begin{multicols}{2}
\begin{enumerate}
\item 2
\item 4
\item 6
\item 10
\end{enumerate}
\end{multicols}
\item Let $S$ be the set of all solutions of the equation $\cos^{-1}{2x} -2 \cos^{-1}{\sqrt{1-x^2}}=\pi,  x \in \sbrak{\frac{-1}{2},\frac{1}{2}} $ Then $\displaystyle \sum_{x\in S} 2 \sin^{-1}\brak{x^2-1}$ is equal to
\begin{multicols}{2}
\begin{enumerate}
\item 0
\item $\frac{-2\pi}{3}$
\item $\pi-\sin^{-1}\frac{\sqrt 3}{4}$
\item $\pi-2\sin^{-1}\frac{\sqrt 3}{4}$
\end{enumerate}
\end{multicols}
\item If the center and radius of the circle $\abs{\frac{z-2}{z-3}}=2$ are respectively $\brak{\alpha ,\beta}$ and $\gamma$, then $3\brak{\alpha +\beta +\gamma}$ is equal to 
\begin{multicols}{2}
\begin{enumerate}
\item 11
\item 9
\item 10
\item 12
\end{enumerate}
\end{multicols}
\item If $y=y\brak{x}$ is the solution curve of the differential equation $\frac{dy}{dx}+y\tan{x}=x\sec{x}, 0\le x\le \frac{\pi}{3},y\brak{0}=1,$ then $y\brak{\frac{\pi}{6}}$  is equal to
\begin{enumerate}
\item $\frac{\pi}{12}-\frac{\sqrt 3}{2}\log_e\brak{\frac{2}{e\sqrt 3}}$
\item $\frac{\pi}{12}+\frac{\sqrt 3}{2}\log_e\brak{\frac{2\sqrt 3}{e}}$
\item $\frac{\pi}{12}-\frac{\sqrt 3}{2}\log_e\brak{\frac{2\sqrt 3}{e}}$
\item $\frac{\pi}{12}+\frac{\sqrt 3}{2}\log_e\brak{\frac{2}{e\sqrt 3}}$
\end{enumerate}
\item Let $R$ be a relation on $\mathbb{R}$, given by $R=\cbrak{\brak{a,b}: 3a-3b+\sqrt 7 \text {is an irrational number}}.$ Then $R$ is 
\begin{enumerate}
\item Reflexive but neither symmetric nor transitive 
\item Reflexive and transitive but not symmetric 
\item Reflexive and symmetric but not transitive 
\item An equivalence relation  
\end{enumerate}	
\end{enumerate}
\end{document}
