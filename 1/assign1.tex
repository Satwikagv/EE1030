%iffalse
\let\negmedspace\undefined
\let\negthickspace\undefined
\documentclass[journal,12pt,twocolumn]{IEEEtran}
\usepackage{cite}
\usepackage{amsmath,amssymb,amsfonts,amsthm}
\usepackage{algorithmic}
\usepackage{graphicx}
\usepackage{textcomp}
\usepackage{xcolor}
\usepackage{txfonts}
\usepackage{listings}
\usepackage{enumitem}
\usepackage{mathtools}
\usepackage{gensymb}
\usepackage{comment}
\usepackage[breaklinks=true]{hyperref}
\usepackage{tkz-euclide} 
\usepackage{listings}
\usepackage{gvv}                                        
%\def\inputGnumericTable{}                                 
\usepackage[latin1]{inputenc}                                
\usepackage{color}                                            
\usepackage{array}                                            
\usepackage{longtable}                                       
\usepackage{calc}                                             
\usepackage{multirow}                                         
\usepackage{hhline}                                           
\usepackage{ifthen}                                           
\usepackage{lscape}
\usepackage{tabularx}
\usepackage{array}
\usepackage{float}


\newtheorem{theorem}{Theorem}[section]
\newtheorem{problem}{Problem}
\newtheorem{proposition}{Proposition}[section]
\newtheorem{lemma}{Lemma}[section]
\newtheorem{corollary}[theorem]{Corollary}
\newtheorem{example}{Example}[section]
\newtheorem{definition}[problem]{Definition}
\newcommand{\BEQA}{\begin{eqnarray}}
\newcommand{\EEQA}{\end{eqnarray}}
\newcommand{\define}{\stackrel{\triangle}{=}}
\theoremstyle{remark}
\newtheorem{rem}{Remark}

% Marks the beginning of the document
\begin{document}
\bibliographystyle{IEEEtran}
\vspace{3cm}

\title{8.circle}
\author{ee24btech11027- satwikagv}
\maketitle
\newpage
\bigskip

\renewcommand{\thefigure}{\theenumi}
\renewcommand{\thetable}{\theenumi}
\begin{enumerate}
\item[6.]The lines $2x-3y=5$ and $3x-4y=7$ are diameters of a circle having area as 154 sq.units.Then the equation of the circle is

\hfill{\textbf{(2003)}}

\begin{enumerate}
\item[(a)] $x^2+y^2-2x+2y=62$
\item[(b)] $x^2+y^2+2x-2y=62$
\item[(c)] $x^2+y^2+2x-2y=47$
\item[(d)] $x^2+y^2-2x+2y=47$
\end{enumerate}
\end{enumerate}
\begin{enumerate}
\item[7.]If a circle passes through the point \brak{a,b} and cuts the circle $x^2+y^2=4$ orthogonally, then the locus of its centre is

\hfill{\textbf{(2004)}}

\begin{enumerate}
\item[(a)] $2ax-2by-\brak{a^2+b^2+4}=0$
\item[(b)] $2ax+2by-\brak{a^2+b^2+4}=0$
\item[(c)] $2ax-2by+\brak{a^2+b^2+4}=0$
\item[(d)] $2ax+2by+\brak{a^2+b^2+4}=4$
\end{enumerate}
\end{enumerate}
\begin{enumerate}
\item[8.]A variable circle passes through the fixed point $\vec{A}\brak{p,q}$ and touches $x$-axis.The locus of the other end of the diameter through $A$is

\hfill{\textbf{(2004)}}

\begin{enumerate}
\item[(a)] $\brak{y-q}^2=4px$
\item[(b)] $\brak{x-q}^2=4py$ 
\item[(c)] $\brak{y-p}^2=4qx$
\item[(d)] $\brak{x-p}^2=4qy$
\end{enumerate}
\end{enumerate}
\begin{enumerate}
\item[9.]If the lines $2x+3y+1=0$ and $3x-y-4=0$ lie along diameter of a circle of circumference $10\pi$, then the equation of the circle is 

\hfill{\textbf{(2004)}}

\begin{enumerate}
\item[(a)] $x^2+y^2+2x-2y-23=0$
\item[(b)] $x^2+y^2-2x-2y-23=0$
\item[(c)] $x^2+y^2+2x+2y-23=0$
\item[(d)] $x^2+y^2-2x+2y-23=0$
\end{enumerate}
\end{enumerate}
\begin{enumerate}
\item[10.]Intercept on the line $y=x$ by the circle $x^2+y^2-2x=0$ is $\vec{AB}$.Equation of the circle on $\vec{AB}$ as a diameter is 

\hfill{\textbf{(2004)}}

\begin{enumerate}
\item[(a)] $x^2+y^2+x-y=0$
\item[(b)] $x^2+y^2-x+y=0$
\item[(c)] $x^2+y^2+x+y=0$
\item[(d)] $x^2+y^2-x-y=0$
\end{enumerate}
\end{enumerate}
\begin{enumerate}
\item[11.]If the circles $x^2+y^2+2ax+cy+a=0$ and $x^2+y^2-3ax+dy-1=0$ intersect in two distinct points $\vec{P}$ and $\vec{Q}$ then the line $5x+by-a=0$ passes through $\vec{P}$ and $\vec{Q}$ for

\hfill{\textbf{(2005)}}

\begin{enumerate}
\item[(a)] exactly one value of $a$
\item[(b)] no value of $a$
\item[(c)] infinitely many values of $a$
\item[(d)] exactly two values of $a$
\end{enumerate}
\end{enumerate}
\begin{enumerate}
\item[12.]A circle touches the $x$-axis and also touches the circle with centre at \brak{0,3} and radius 2. The locus of the centre of the circle is

\hfill{\textbf{(2005)}}

\begin{enumerate}
\item[(a)] an ellipse
\item[(b)] a circle 
\item[(c)] a hyperbola
\item[(d)] a parabola
\end{enumerate}
\end{enumerate}
\begin{enumerate}
\item[13.]If a circle passes through the point \brak{a,b} and cuts the circle $x^2+y^2=p^2$ orthogonolly, then the equation of the locus of its centre is 

\hfill{\textbf{(2005)}}

\begin{enumerate}
\item[(a)] $x^2+y^2-3ax-4by+\brak{a^2+b^2-p^2}=0$
\item[(b)] $2ax+2by-\brak{a^2-b^2+p^2}=0$
\item[(c)] $x^2+y^2-2ax-3by+\brak{a^2-b^2-p^2}=0$
\item[(d)] $2ax+2by-\brak{a^2+b^2+p^2}=0$
\end{enumerate}
\end{enumerate}
\begin{enumerate}
\item[14.]If the pair of lines $ax^2+2(a+b)xy+by^2=0$ lie along diameters of a circle and divide the circle into four sectors such that the area of one of the sectors is thrice the area of another sector then

\hfill{\textbf{(2005)}}

\begin{enumerate}
\item[(a)] $3a^2-10ab+3b^2=0$
\item[(b)] $3a^2-2ab+3b^2=0$
\item[(c)] $3a^2+10ab+3b^2=0$
\item[(d)] $3a^2+2ab+3b^2=0$
\end{enumerate}
\end{enumerate}
\begin{enumerate}
\item[15.]If the lines $3x-4y-7=0$ and $2x-3y-5=0$ are two diameters of a circle of area $49\pi$ square units, the equation of the circle is 

\hfill{\textbf{(2006)}}

\begin{enumerate}
\item[(a)] $x^2+y^2+2x-2y-47=0$
\item[(b)] $x^2+y^2+2x-2y-62=0$
\item[(c)] $x^2+y^2-2x+2y-62=0$
\item[(d)] $x^2+y^2-2x+2y-47=0$
\end{enumerate}
\end{enumerate}
\begin{enumerate}
\item[16.]Let $\vec{C}$ be the circle with centre \brak{0,0} and radius 3 units.The equation of the locus of the mid points of the chords of the circle $\vec{C}$ that subtend an angle of $\frac{2\pi}{3}$ at its centre is

\hfill{\textbf{(2006)}}

\begin{enumerate}
\item[(a)] $x^2+y^2=\frac{3}{2}$
\item[(b)] $x^2+y^2=1$
\item[(c)] $x^2+y^2=\frac{27}{4}$
\item[(d)] $x^2+y^2=\frac{9}{4}$
\end{enumerate}
\end{enumerate}
\begin{enumerate}
\item[17.]Consider a family of circles which are passing through the point \brak{-1,1}, and are tangent to $x$-axis.If \brak{h,k} are the coordinate of the centre of the circles, then the set of values of $k$ is given by the interval

\hfill{\textbf{(2007)}}

\begin{enumerate}
\item[(a)] $\frac{-1}{2} \le k \le \frac{1}{2}$
\item[(b)] $k \le \frac{1}{2}$
\item[(c)] $o \le k \le \frac{1}{2}$
\item[(d)] $k \ge \frac{1}{2}$
\end{enumerate}
\end{enumerate}
\begin{enumerate}
\item[18.]The point diametrically opposite to the point $\vec{P}\brak{1,0}$ on the circle $x^2+y^2+2x+2y-3=0$ is 

\hfill{\textbf{(2008)}}

\begin{enumerate}
\item[(a)] \brak{3,-4}
\item[(b)] \brak{-3,4}
\item[(c)] \brak{-3,-4}
\item[(d)] \brak{3,4}
\end{enumerate}
\end{enumerate}
\begin{enumerate}
\item[19.]The differential equation of the family of circles with fixed radius 5 units and centre on the line $y=2$ is


\begin{enumerate}
\item[(a)] $\brak{x-2}y'^2=25-\brak{y-2}^2$
\item[(b)] $\brak{y-2}y'^2=25-\brak{y-2}^2$
\item[(c)] $\brak{y-2}^2y'^2=25-\brak{y-2}^2$
\item[(d)] $\brak{x-2}^2y'^2=25-(y-2)^2$
\end{enumerate}
\end{enumerate}
\begin{enumerate}
\item[20.]If P and Q are the points of intersection of the circles $x^2+y^2+3x+7y+2p-5=0$ and $x^2+y^2+2x+2y-p^2=0$ then there is a circle passing through $\vec{P}$,$\vec{Q}$ and \brak{1,1} for:

\hfill{\textbf{(2009)}}

\begin{enumerate}
\item[(a)] all except one value of $p$
\item[(b)] all except two values of $p$
\item[(c)] exactly one value of $p$
\item[(d)] all value of $p$
\end{enumerate}
\end{enumerate}
\end{document}
